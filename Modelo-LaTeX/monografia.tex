%%
%% Arquivo principal
%%

\documentclass[letterpaper,12pt,openright,twoside]{book}

\usepackage[latin1]{inputenc}
\usepackage[T1]{fontenc}
\usepackage{times}
\usepackage[portuges, brazil]{babel}    % hiphena��o em portugues
\usepackage{indentfirst}        % indenta primeiro par�grafo
\usepackage{graphicx}
\usepackage{fancyhdr}
\usepackage{wasysym}
\usepackage{pifont}
\usepackage{textcomp}      % \texttrademark
\usepackage{nomencl}    % glossario
\usepackage{url}  
\usepackage{multirow}  
%\usepackage[round]{natbib}  % cita��es tipo  (nome-ano)
%\usepackage{natbib}  % cita��es tipo  [nome-ano]
\usepackage[alf]{abntcite} % cita��es no formato abnt
\usepackage{setspace}

%% tamanho do texto e margens
\headheight 16pt
\setlength{\topmargin}{-15pt} % extra vert. space + at the top of header: 23pt
\setlength{\oddsidemargin}{0pt} % extra spc added at the left of odd page: 0pt
\setlength{\evensidemargin}{-12pt} % ext. spc added at the left of even pg: 59pt
\setlength{\textheight}{638pt} % height of the body: 592pt
\setlength{\textwidth}{483pt} % width of the body: 470pt

%% Estilo da p�gina corrente e demais p�ginas
\pagestyle{fancyplain}
\renewcommand{\chaptermark}[1]{\markboth{#1}{}}
\renewcommand{\sectionmark}[1]{\markright{\thesection\ #1}}
\lhead[\fancyplain{}{\bfseries\thepage}]{\fancyplain{}{\bfseries\rightmark}}
\rhead[\fancyplain{}{\bfseries\leftmark}]{\fancyplain{}{\bfseries\thepage}}
\cfoot[\fancyplain{\bfseries\thepage}{}]{\fancyplain{\bfseries\thepage}{}}

% Lista de arquivos a serem processados
\includeonly{capa,resumos2,Cap1/cap1,ApendiceA/apA}

%% Para a cria��o do Gloss�rio
\makenomenclature

%% Comandos para facilitar a edi��o 
\newcommand{\cpp}{\texttt{C$++$}}
\newcommand{\latex}{\LaTeX}


%% Inicia o texto
\begin{document}

%% Abrevia figuras e tabelas
\def\figurename{Fig.}
\def\tablename{Tab.}

%% Paginas iniciais (sem numera��o)
\frontmatter

%% P�gina de rosto
%%
%% ********** P�gina de Rosto
%%

%% n�o numerar a p�gina
\thispagestyle{empty}

\begin{center}
{\large Nome da Disciplina ou Autor}
 
\vspace{0.9in}
{\Large \textbf{Primeira Linha do T�tulo}} \\ \vspace{2ex}
{\Large \textbf{Segunda Linha do T�tulo}} \\

\vspace{2in}
\begin{flushright}
\parbox{3.50in}{
Informa��es adicionais}
\end{flushright}

\vspace{0.2in}
\begin{flushright}
\parbox{3.50in}{Autor 1 \\ 
Autor 2}
\end{flushright}


\vspace{1.2in}
Campinas, SP \\
2009
\end{center}





%% Resumo/Abstract
\include{resumos2}

%% Lista de conte�do (sum�rio)
\def\contentsname{Sum�rio}
\tableofcontents

%% Lista de figuras (gerada automaticamente)
\cleardoublepage
\addcontentsline{toc}{chapter}{Lista de Figuras}
\listoffigures

%% Lista de tabelas (gerada automaticamente)
\cleardoublepage
\addcontentsline{toc}{chapter}{Lista de Tabelas}
\listoftables

%% Gloss�rio (gerado automaticamente - veja entradas em cap1.tex)
\cleardoublepage
\renewcommand{\nomname}{Gloss�rio}
\markboth{GLOSS�RIO}{GLOSS�RIO}
\addcontentsline{toc}{chapter}{\nomname}
\printnomenclature

\mainmatter

%% Cap. 1 - Introdu��o
%%
%% Cap�tulo 1: Modelo de Cap�tulo
%%
\chapter{Modelo de cap�tulo}
\label{Cap:modelo}

O cap�tulo deve conter uma introdu��o e um fecho.
A introdu��o fornece ao leitor uma breve descri��o
do que ser� tratado no cap�tulo, enquanto o fecho
apresenta coment�rios finais sobre o que foi desenvolvido
no cap�tulo. 

Cap�tulos s�o divididos em se��es. O n�mero ideal de
se��es � imposs�vel de se precisar. Entretanto, um
cap�tulo com uma �nica se��o provavelmente deve ser
agregado ao cap�tulo anterior ou posterior. Um cap�tulo
com quinze se��es provavelmente deve ser subdividido em
dois cap�tulos.

Cap�tulos, se��es e subse��es devem ser rotulados para que 
possam ser  referenciados em qualquer parte do texto. 
Exemplo: $\ldots$ no cap�tulo~\ref{Cap:modelo} apresentamos
um modelo de cap�tulo de tese.


\section{Se��es}
\label{Sec:secoes}  %% labels n�o devem conter caracteres acentuados

Se��es s�o divis�es do conte�do do cap�tulo. Esta divis�o
deve ser l�gica (tem�tica) e n�o f�sica (por tamanho).
Por exemplo, um cap�tulo que trata de \textit{software}
de sistema teria se��es que tratam de montadores, ligadores,
carregadores, compiladores e sistemas operacionais.

Tal como cap�tulos, se��es devem ser rotuladas para refer�ncia
em outras partes do texto. Se��es s�o divididas em subse��es.

\subsection{Subse��es}
\label{Sec:subsecoes}

Subse��es s�o divis�es de se��es. No exemplo do texto sobre \textit{software}
de sistema, a se��o re\-fe\-ren\-te a sistema operacional conteria, 
por exemplo, subse��es que tratam de arquivos, processos, mem�ria e 
entrada/sa�da. Tal como se��es, subse��es s�o divis�es tem�ticas do texto.


\subsection*{Subsubse��es}
\label{Sec:subsubsecoes}

Subsubse��es s�o divis�es de subse��es e n�o devem ser numeradas no texto.
O $\ast$ ap�s o comando \textit{subsubsection} instrui o \latex~a n�o numerar
a subsubse��o.


\section{Figuras, tabelas e gr�ficos}
\label{Sec:figuras}

Figuras s�o editadas com editores gr�ficos capazes de exportar
a figura em formatos PS (\textit{PostScript}) 
ou, preferencialmente, EPS (\textit{Encapsulated PostScript}).
O editor \textit{xfig} � adequado para a maioria dos casos.
Por exemplo, a figura~\ref{fig:PNO-PrvSrv} foi editada no
\textit{xfig}. Outra op��o � o editor {\texttt dia}, um editor 
orientado a diagramas (UML, fluxograma, etc.) com capacidade de exportar 
EPS~\cite{hp-dia03}  Figuras em formato GIF, JPEG e bitmap podem ser
convetidas para o formato EPS atrav�s do aplicativo 
{\textit xv}. {\textit xv} n�o lista o formato EPS
dentre aqueles que � capaz de manipular. Entretanto,
selecionando-se o formato \textit{PostScript} e fornecendo-se
a extens�o {\textit .eps} ao nome do arquivo, o formato
EPS � gerado.


\begin{figure}[!hbt]
\centering \includegraphics[width=12cm]{{./Cap1/pno-retl.eps}}
\caption{Operador de rede - provedor de servi�o.}
\label{fig:PNO-PrvSrv}
\end{figure}

Tabelas s�o constru�das com comandos pr�prios do \latex. Por
exemplo, a tabela~\ref{tab:Cmpt-SessAcss-ArchServ} foi constru�da desta forma.

\begin{table}[!htb]  \begin{center} 
\begin{tabular}{|c|c|l|c|} \hline\hline
\textit{Categoria} & \textit{Dom�nio} & \textit{Componente} & \textit{Acr�nimo}
 \\ \hline\hline
\multirow{5}{4cm}{Sess�o de Acesso} & \multirow{2}{2cm}{Usu�rio} &
 Aplica��o do Usu�rio & asUAP \\ \cline{3-4}
  & & Agente do Provedor   & PA \\ \cline{2-4}
  & \multirow{3}{2cm}{Provedor} & Agente do Usu�rio & UA \\ \cline{3-4}
  & & Agente de Usu�rio Especificado & namedUA \\ \cline{3-4}
  & & Agente de Usu�rio An�nimo & anonUA \\ \hline\hline
\end{tabular} 
\end{center}
\caption{Componentes da sess�o de acesso da arquitetura de servi�o TINA.}
\label{tab:Cmpt-SessAcss-ArchServ}
\end{table}


Gr�ficos s�o gerados com aplicativos capazes de exportar
nos formatos PS ou EPS. A ferramenta \textit{gnuplot} �
uma das mais utilizadas para a gera��o de gr�ficos. 
Uma vez no formato EPS, gr�ficos s�o inseridos no texto
tal como figuras (vide figura~\ref{fig:PbmDiversos}).


\begin{figure}[!hbt]
\centering \includegraphics[width=.7\linewidth]{./Cap1/PbmDiversos.eps}
\caption{$P_b$; desvanecimentos arbitr�rios~\mbox{($M=3$).}}
\label{fig:PbmDiversos}
\end{figure}


\section{Cita��es bibliogr�ficas}
\label{Sec:citacoes}

\latex~utiliza um arquivo em separado para as refer�ncias bibliogr�ficas.
Este arquivo deve possuir o mesmo nome do arquivo mestre com a extens�o \textit{bib}.
Arquivos \textit{bib} possuem os seguintes estilos de refer�ncia:
\begin{itemize} \itemsep -0.5ex
\item artigos em anais de simp�sios~\cite{proc-faina01};
\item artigos em colet�neas de artigos~\cite{col-pinto00};
\item cap�tulos de livros~\cite{inbook-santos00};
\item anais de simp�sios~\cite{proc-sbrc02};
\item livros~\cite{book-lamport94};
\item teses de doutorado~\cite{phd-faina01};
\item teses de mestrado~\cite{msc-santos03};
\item relat�rios t�cnicos~\cite{rep-omg2000};
\item manuais t�cnicos~\cite{man-orbix99};
\item trabalhos n�o publicados~\cite{unp-sichman02};
\item p�ginas na Internet~\cite{hp-dia03} (utilizar como data a data do �ltimo acesso � p�gina);
\item miscel�nea~\cite{misc-cruz03}.
\end{itemize}


\section{Equa��es}
\label{Sec:equacoes}

\latex~� insuper�vel no processamento de equa��es. Equa��es
simples como $2^{n}$ podem ser editadas no pr�prio texto.
Equa��es complexas como


\begin{eqnarray} \label{eq:PDF:RSR}
  p \left( \gamma \right) & = & \frac{1}{2} \sqrt{\frac{M}{\gamma \bar{\gamma}_{b}}} \frac{1}{ \prod_{i=1}^M {\sqrt{\tilde{\gamma}_i}}}
  \int_0^{\sqrt{M \delta}} \int_0^{\sqrt{M \delta} - r_M } \cdots
  \int_0^{\sqrt{M \delta} - \sum_{i = 3}^M {r_i } } \nonumber \\
  & & p \left( {\frac{\sqrt{M \delta} - \sum_{i = 2}^M {r_i }}{\sqrt{\tilde{\gamma}_1}} ,
  \frac{r_2}{\sqrt{\tilde{\gamma}_2}} , \ldots ,\frac{r_M}{\sqrt{\tilde{\gamma}_M}} } \right)
  \, dr_2 \cdots dr_{M-1} \, dr_M
\end{eqnarray}

ou


\begin{equation}\label{eq:TrCGI}
  T(r) = \frac{1}{f_m}
  \left( \frac{\pi}{2} \sum_{i=1}^M
  {\tilde{r}_i^2 \dot{\varsigma}_i^2}\right)^{-1/2}
  \frac
  {\begin{array}{ll}
  \int_0^{\rho \sqrt{M}} \int_0^{\rho \sqrt{M} - r_M } \cdots
  \int_0^{\rho \sqrt{M} - \sum_{i = 3}^M {r_i } } \int_0^{\rho \sqrt{M} -
  \sum_{i = 2}^M {r_i } }  \\
  p \left( {\frac{r_1}{\tilde{r}_1} ,
  \frac{r_2}{\tilde{r}_2} , \ldots ,\frac{r_M}{\tilde{r}_M} } \right)
  \, dr_1 \, dr_2 \cdots dr_{M-1} \, dr_M \\ \end{array}}
  {\begin{array}{ll}
  \int_0^{\rho \sqrt{M}} \int_0^{\rho \sqrt{M} - r_M } \cdots
  \int_0^{\rho \sqrt{M} - \sum_{i = 3}^M {r_i } } \\
  p \left( {\frac{\rho \sqrt{M} - \sum_{i = 2}^M {r_i }}{\tilde{r}_1} ,
  \frac{r_2}{\tilde{r}_2} , \ldots ,\frac{r_M}{\tilde{r}_M} } \right)
  \, dr_2 \cdots dr_{M-1} \, dr_M \\ \end{array}}
\end{equation}

\noindent
s�o automaticamente numeradas e podem ser referenciadas a partir do
texto. Por exemplo, a equa��o~\ref{eq:TrCGI} � trivialmente derivada
da equa��o~\ref{eq:PDF:RSR}.



\section{Gloss�rio}
\label{Sec:glossario}

\latex~gera automaticamente o gloss�rio. Ao redigir uma sigla pela primeira vez, o
autor, ao final do par�grafo, gera a entrada para o gloss�rio. Exemplo:

A camada de adapta��o tipo 1 (ATM AAL1) das redes digitais de servi�os integrados em
faixa larga (B-ISDN) � especificada pela ITU-T.
\nomenclature{ATM - }{Asynchronous Transfer Mode}
\nomenclature{AAL1 - }{ATM Adaptation Layer Type 1}
\nomenclature{B-ISDN - }{Broadband Integrated Service Digital Network}
\nomenclature{ITU-T - }{International Telecommunication Union-Telecommunication Standardization Sector}


\section{Estilo}
\label{Sec:estilo}

Recomenda-se espa�amento 1.5 entre as linhas do texto.
Evite sempre os seguintes recursos (ou melhor, enfeites):
\begin{itemize}
\item \textbf{o uso de negrito;}
\item \textit{o uso de it�lico (exceto em palavras em outra l�ngua);}
\item \texttt{texto em diferente fonte como m�quina de escrever;}
\item \underline{o uso de texto sublinhado;}
\item o uso excessivo de~\footnote{Notas de rodap�.}.
\end{itemize}

Lembre-se: um texto ``limpo'' � mais agrad�vel de ler que um texto ``enfeitado''.




%% Refer�ncias bibliog�ficas (geradas automaticamente)
\addcontentsline{toc}{chapter}{Refer�ncias bibliogr�ficas}
%\bibliographystyle{plainnat}  %% nome-ano
%\bibliographystyle{unsrt}    %% numero
\bibliographystyle{abnt-alf}  %% nome-ano
\bibliography{monografia}

\appendix
%%Ap�ndice A 
\chapter{Ap�ndices}

No \latex~ap�ndices s�o editados como cap�tulos. O comando \textit{apendix} 
(vide arquivo mestre) faz com que todos os cap�tulos seguintes sejam
considerados ap�ndices.

Ap�ndices complementam o texto principal da tese com informa��es
para leitores com especial interesse no tema, devendo ser considerados
leitura opcional, ou seja, o entendimento do texto principal da tese
n�o deve exigir a leitura atenta dos ap�ndices.

Ap�ndices usualmente contemplam provas de teoremas, dedu��es 
de f�rmulas matem�ticas, diagramas esquem�ticos, gr�ficos e
trechos de c�digo. Quanto a este �ltimo, c�digo extenso n�o
deve fazer parte da tese, mesmo como ap�ndice. O ideal
� disponibilizar o c�digo na Internet para os interessados
em examin�-lo ou utiliz�-lo.


\end{document}

