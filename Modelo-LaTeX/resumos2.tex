%%
%% ********** Resumo
%%
\newpage \thispagestyle{plain} 
\vspace{1.5cm}
\begin{center}
{\huge{\textbf{Resumo}}}
\end{center}
\vspace{0.5cm}

O resumo deve apresentar ao leitor uma id�ia compacta, mas clara
do trabalho descrito na tese. A defini��o precisa e import�ncia do problema
abordado, os principais objetivos, motiva��es e desafios da pesquisa s�o bons pontos 
de partida para o resumo. A estrat�gia ou metodologia empregada 
na pesquisa, suas principais contribui��es e os resultados
mais importantes tamb�m devem fazer parte do resumo. Note que tanto 
o resumo quanto o {\it abstract} devem compartilhar a mesma p�gina.

\vspace{1.5ex}

{\bf Palavras-chave}: Processamento de texto, \latex,
Prepara��o de Teses, Relat�rios T�cnicos.

%%
%% ********** Abstract
%%

\vspace{1.5cm}
\begin{center}
{\huge{\textbf{Abstract}}}
\end{center}
\vspace{0.5cm}

The abstract must present to the reader a short, but clear idea
of the work being reported in the thesis. The precise definition
and importance of the problem being addressed, the main objectives,
motivations and challenges of the research are a good starting point for the
abstract. The strategy or metodology employed
in the research, its main contributions, and the
most important results achieved may be part of the abstract as well. Notice 
that the {\it resumo} and the abstract must share the same
page.

\vspace{1.5ex}

{\bf Keywords}: Document Processing, \latex, Thesis Preparation,
Technical Reports.



