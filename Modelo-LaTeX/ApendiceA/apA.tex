\chapter{Ap�ndices}

No \latex~ap�ndices s�o editados como cap�tulos. O comando \textit{apendix} 
(vide arquivo mestre) faz com que todos os cap�tulos seguintes sejam
considerados ap�ndices.

Ap�ndices complementam o texto principal da tese com informa��es
para leitores com especial interesse no tema, devendo ser considerados
leitura opcional, ou seja, o entendimento do texto principal da tese
n�o deve exigir a leitura atenta dos ap�ndices.

Ap�ndices usualmente contemplam provas de teoremas, dedu��es 
de f�rmulas matem�ticas, diagramas esquem�ticos, gr�ficos e
trechos de c�digo. Quanto a este �ltimo, c�digo extenso n�o
deve fazer parte da tese, mesmo como ap�ndice. O ideal
� disponibilizar o c�digo na Internet para os interessados
em examin�-lo ou utiliz�-lo.
